\section{Field Report}

\subsection{Konrad Hoepli}

As an enthusiast for new technologies, graphical technologies have always seemed very interesting to me, but the development of a graphical component itself has continuously been avoided until very recently since I pretty much feared the complexity it is said to incorporate. This project occurred to me as the perfect opportunity to stand up to this fear and gather my first experiences with it.
I was definitely not disappointed by the complexity in the tutorial we followed in the beginning of the project to get familiar with OpenGL, but managed to understand it rather well to my surprise. The fact that this technology is commonly used together with \verb|C++| and I would therefore have to step out of my comfort zone even more admittedly was not the way I would have liked, but this also ensured an even bigger learning experienced in the end.

I have definitely extended my knowledge in the realm of OpenGL more than I have in \verb|C++|, but learned to appreciate both of them especially when combined with powerful external libraries. Even mathematical foundations that were very difficult to understand at first ended up being easily implemented and extremely performant.
There have been a lot of difficult problems and mistakes made over the course of this project and as my long illness indicates, it has not been easy to cope with at times. The amount of uncertainties and stress I got myself into in combination with an unhealthy mentality simply overwhelmed me in the end and it took some time to get a decent hold of myself again.
Nevertheless I am quite satisfied with this project, even if it is to be interpreted more as a learning experience than a project that exceeded our expectations.

I am very glad to have had Roberto as my teammate and very impressed by his dedication and understanding. The interaction with and support by Prof. Augenstein also was a very positive factor in every aspect of this project once more and I really am extremely thankful for all he has done for us and the outcome of this project.


\subsection{Roberto Cuervo}

During this project new or only in theory known technologies like OpenGL or WindowsForms come up to me. Learning them was exciting, and I can affirm without doubt that I learned a lot. As usual in these studies, a lot of new doors opened to me.

The more detailed work with \verb|C++| allowed me to learn more specific features of this programming language and to appreciate how powerful it is despite of its complexity. And to realize, that I only have seen a fraction of its possibilities, being the same with OpenGL.

It is said that you learn more from bad experiences. In my case I found several points to improve, related to improve the communication with the client and improve the project planning, among others.

The collaboration with Konrad H\"opli was again a pleasure, and I have not enough words to thank Prof. Augenstein for his support.