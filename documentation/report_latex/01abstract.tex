\section*{Acknowledgements}

We would like to thank specially to Prof. Augenstein for his support during the whole bachelor thesis. It was a luxury for us having him as advisor.

Also we thank the Clinic of Masticatory Disorders for its kindness and availability. For the help from the Assistants Thomas Corbat and Lukas Kretschmar we are grateful too.
\newpage


\section{Abstract}
	\subsection{Introduction}
	
	The Clinic of Masticatory Disorders offers treatment for facial pain and mandibular joint problems. For better diagnosis, the clinic has developed a proprietary  3D  camera  system  (Optis)  to  record  the  patient's  mastication movement. They are also able to extract bones from an MRI image of a patient and  store  them  as  Stereo  Lithography  (STL)  Files.  In  another  proprietary software  called  TMJViewer,  these  resources  can  be  merged  to display  the movement of the teeth and bones in a 3D animation, which is used for medical analysis. 
	
	The current process to achieve this goal requires multiple systems, time consuming manual input, and is therefore error-prone. In addition, the current solution does not allow to display movement data in real time, which complicates the diagnosis process, because immediate feedback to the patient is not possible. 
	
	The goal of this thesis is to develop a single, OpenGL based application that simplifies the above process and further allows real time data analysis.
	
	\subsection{Approach}
	
	To get familiar with OpenGL as well as various aspects of graphics rendering, we consumed some literature on the topic and walked through a tutorial that implemented Phong shading in OpenGL. We then used this code as a base for our project and extended its functionality according to requirements and thus obtained a working C++ prototype for our project. For that  we  imported  anatomical  objects  (STL  files)  and  transferred  them  to OpenGL  as  vertex buffer  objects.  
	
	In  a  second  step  we  were  then  using movement information from Optis to calculate rigid transformations which were afterwards applied to vertex data within a vertex shader. Finally we refactored the  prototype  before integrating  it  into  a  new  WindowsForms  application providing a graphical user interface for its configuration. 
	
	Due to an incomplete specification  of  the  Optis  calibration  process,  the  displayed  movement  ended up not being completely accurate and a lot of reverse engineering was required to reach the resulting version of our software.
	
	\subsection{Results}
	
	This project resulted in an application consisting of a library component for the graphical display embedded into a minimalistic GUI for configuration. As the functionality of the existing application is not yet fully implemented, the software is not suited for productive use. However, it serves as a decent base for a replacement of the existing application and can be used - in combination with our documentation - as a solid introduction to OpenGL.